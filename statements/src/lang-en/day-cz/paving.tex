\input sys/inputs.tex

\begin{document}

\bigheading{Square Paving}

% \info{task_name}{infile}{outfile}{points}{timelimit}{memlimit}
% leave this values, if you are not interested
\info{paving}{stdin}{stdout}{100}{2000 ms}{1 GB}

In the Czech city called Kocourkov a historical mosaic-paved square will be
reconstructed. The square is paved with white and gray stones, which are
arranged in a rectangular grid. The preservationists have two requirements for
the reconstruction:

\begin{enumerate}[nolistsep]
\item Each square 2x2 must contain an odd number of gray stones.
\item Some historical stones are protected by law; therefore, they must remain
  in their original location.
\end{enumerate}

The councillors would like to know the number of ways in which the square can be
paved so that both requirements are met. Could you help them with this problem?

\heading{Task}

You are given the size of the square and a list of protected stones. Calculate
the number of ways in which the remaining stones can be chosen so that each
square 2x2 contains an odd number of gray stones.

\heading{Input}

The first line of the input contains three integers $N$, $M$ and $K$: the number
of rows of the grid, the number of columns of the grid and the number of
protected stones.

Then, $K$ more lines describing the protected stones follow. Each contains three
integers $R_i$, $C_i$ and $G_i$: the row and the column where the $i$-th stone
is located and its color (number 0 means white color, number 1 gray color).

\bigskip
\noindent
It holds $2 \leq N,M \leq 10^5$ and $0 \leq K \leq 10^5$.\\
For all $i$ it holds $1 \leq R_i \leq N$, $1 \leq C_i \leq M$ and
  $G_i \in \{ 0, 1 \}$.\\
No two protected stones have equal coordinates.\\
In the 20 \% of testcases $N,M \leq 5$ and $K \leq 5$.\\
In the 40 \% of testcases $N,M \leq 5000$ and $K \leq 25$.

\heading{Output}

The output must contain exactly one line containing the number of ways in which
the square can be paved. Because the resulting number can be very large, output
only its remainder after division by $10^9 + 7$.

\heading{Sample}

\sampleIN
3 4 3
2 2 1
1 2 0
2 3 1
\sampleOUT
8
\sampleEND

\end{document}
