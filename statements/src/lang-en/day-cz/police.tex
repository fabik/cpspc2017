\input sys/inputs.tex

\begin{document}

\bigheading{Police Station}

% \info{task_name}{infile}{outfile}{points}{timelimit}{memlimit}
% leave this values, if you are not interested
\info{police}{stdin}{stdout}{100}{2000 ms}{1 GB}

A network of hyperspace highways is built in the galaxy. Each of the highways
is a one-directional corridor which connects two planets. Galactic government
wants to find a planet on which a police station will be built.

In order for the police to protect the whole galaxy, it must be possible to
travel from the police station to every planet in the galaxy using the
hyperspace highways network.

It is not necessary that the police can return back to the police station from
any planet using the hyperspace highways. The police needs not to hurry on the
way back, and so they can use slower ways than highways.

The government does not know how to deal with finding the suitable planet for
the police station; therefore, they ask for your help.

\heading{Task}

You are given the network of one-directional hyperspace highways. Find all
planets from which it is possible to reach all other planets in the galaxy using
some sequence of highways.

\heading{Input}

The first line of the input contains two integers $N$ a $M$: the number of
planets and the number of highways.

Then, $M$ more lines follow. Each contains two integers $A_i$ and $B_i$: the
numbers of planets connected by $i$-th highway. Each of the highways is
one-directional and can only be used to travel from the planet $A_i$ to the
planet $B_i$.

\bigskip
\noindent
It holds $1 \leq N, M \leq 10^6$.\\
For all $i$ it holds $1 \leq A_i, B_i \leq N$ and $A_i \neq B_i$.\\
No two highways connect equal planets in the same direction. However, they can
  connect equal planets in the opposite direction.\\
Furthermore, in 30 \% of the testcases $N \leq 10^3$.

\heading{Output}

Output two lines. The first line should contain the total number of planets
which are suitable for the police station. The second line should contain
a space-separated list of these planets in ascending order.

\heading{Sample}

\sampleIN
5 6
1 3
1 4
4 2
2 1
2 5
5 4
\sampleOUT
4
1 2 4 5
\sampleEND

\sampleIN
3 2
1 3
2 3
\sampleOUT
0
~
\sampleEND

\end{document}
