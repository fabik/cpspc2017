\input sys/inputs.tex

\begin{document}

\bigheading{Robot Race}

% \info{task_name}{infile}{outfile}{points}{timelimit}{memlimit}
% leave this values, if you are not interested
\info{race}{stdin}{stdout}{100}{2000 ms}{1 GB}

A robot race on a maze will be held in Byteland. The maze has a shape of
a rectangle and is divided into $n \times m$ fields arranged in $n$ rows and $m$
columns. Each of the fields is either empty or contains an obstacle.

The contestants register their robots for the competition and come to the maze
one after another. Each of the contestants gets random coordinates of the
initial field and the target field. The robot is then placed in the initial
field and must get to the target field using a sequence of steps.

To make the game more challenging, the rules specify that in each step the robot
can only move one field right or one field down in the maze. Moving the robot in
any other direction is not allowed.

The contestant whose robot gets fastest from the initial field to the target
field wins the competition. If the robot does not get to the target field in the
time limit, the contestant in disqualified.

The organizers of the competition realized that if the contestant gets bad
coordinates, the robot will not be able to get to the target field using any
sequence of moves. In that case they would like to give the contestant another
pair of coordinates.

\heading{Task}

You are given a map of a maze of size $n \times m$ and $q$ pairs of coordinates
of the initial and target fields. Determine for each pair of coordinates whether
it is possible to get from the initial field to the target field using
a sequence of steps right or down.

\heading{Input}

The first line contains three integers $n$, $m$ and $q$: the number of rows and
columns of the maze and the number of pairs of coordinates.

Each of the following $n$ lines contains $m$ characters describing the fields of
the maze. The character \texttt{.} represents an empty field and the character
\texttt{\#} a field with an obstacle.

Then $q$ lines follow describing the pairs of coordinates. The $i$-th of them
contains four integers ${r_1}_i$, ${c_1}_i$, ${r_2}_i$, ${c_2}_i$: the row and
the column of the initial field and the row and the column of the target field.

\bigskip
\noindent
It holds $1 \leq n, m \leq 1000$ and $1 \leq q \leq 10^6$.\\
For all $i \in \{ 1, 2, \ldots, q \}$ it holds $1 \leq {r_1}_i, {r_2}_i \leq n$
  and $1 \leq {c_1}_i, {c_2}_i \leq m$.\\
For all $i \in \{ 1, 2, \ldots, q \}$ the fields in the maze with coordinates
  $({r_1}_i, {c_1}_i)$ and $({r_2}_i, {c_2}_i)$ are empty.\\
Furthermore, in 20 \% of the testcases $q \leq 300$.

\heading{Output}

Output $q$ lines. The $i$-th line should contain the word \texttt{YES} if the
robot can get from the $i$-th initial field to the $i$-th target field, or the
word \texttt{NO} otherwise.

\heading{Sample}

\sampleIN
3 4 5
.\#..
.\#\#.
....
1 1 3 4
1 3 3 4
1 1 1 1
1 1 2 4
2 1 2 4
\sampleOUT
YES
YES
YES
NO
NO
\sampleEND

\end{document}
