\input sys/inputs.tex

\usepackage[slovak]{babel}
\usepackage{wrapfig}

\begin{document}

\bigheading{The Battle for Wesnoth}

% \info{task_name}{infile}{outfile}{points}{timelimit}{memlimit}
% leave this values, if you are not interested
\info{wesnoth}{stdin}{stdout}{100}{100 ms}{1 GB}

Poznáte The Battle for Wesnoth? Je to celkom fajná ťahová stratégia s fantasy tématikou. Sú v nej
elfovia, ohyzdi, nemŕtvi, trpaslíci a draky! V tejto úlohe potrebujete vedieť, že každá jednotka má niekoľko životov (HP) a tiež ako v Battle for Wesnoth prebieha boj.

Bude nás zaujímať iba najjednoduchší prípad boja: obyčajný útok na bezbrannú jednotku. Takýto útok
sa dá popísať troma celými číslami:

\begin{itemize}
 \item $d$ -- zranenie (počet stratených HP) spôsobené úspešným úderom;
 \item $b$ -- počet úderov;
 \item $p$ -- pravdepodobnosť (v percentách) že úder je úspešný.
\end{itemize}

Čísla $d$ a $b$ sú v hre vlastnosťami útočníka, zatiaľ čo $p$ je väčšinou určené terénom, na ktorom
stojí brániaca sa jednotka. V našom prípade však budeme predpokladať, že aj $p$ je vlastnosťou útočníka
(napríklad ako pri magických útokoch v BfW).

Napríklad, majme útok s $d=6$, $b=2$, a $p=60$. Tento útok sa môže skončiť troma rôznymi výsledkami:

\begin{itemize}
 \item So šancou $16\,\%$, útočník oba údery netrafí a nespôsobí obrancovi žiadne zranenie.
 \item So šancou $48\,\%$, útočník jeden z úderov trafí a jeden netrafí, spôsobiac obrancovi $6$ zranenia.
 \item So šancou $36\,\%$, útočník trafí oba údery, spôsobiac dokopy $12$ zranenia.
\end{itemize}

Keď počet životov jednotky klesne na nekladné číslo, jednotka umrie.

David práve hrá rozšírenie Wesnothu so špeciálnou jednotkou, Elfskou Princeznou, ktorá má veľmi špeciálny magický útok:
Namiesto parametrov $d$ a $b$ má iba jeden parameter $m$. Kedykoľvek Elfská Princezná zaútočí, hráč
si môže zvoliť ľubovoľné $d$ a $b$ za podmienky, že to budú kladné celé čísla spĺňajúce 
$d \cdot b \leq m$.

David občas potrebuje, aby jeho Elfská Princezná zabila ohyzdného nemŕtveho kostlivca alebo kostnatého
ohyzdieho bojovníka, takže by rád vedel, aký výber $d$ a $b$ mu dá najväčšiu pravdepodobnosť, že
nepriateľa zabije.

\heading{Úloha}

Dostanete parametre Princeznej $p$ a $m$.
Pre každú jednotku, ktorú David potrebuje zabiť, nájdite optimálnu voľbu $d$ a $b$.

\heading{Vstup}

Prvý riadok vstupu obsahuje dve celé čísla $m, p$ ($1\leq m\leq 10^6$, $1\leq p\leq 99$) -- parametre
útočníka. Druhý riadok obsahuje celé číslo $n$ ($1\leq n\leq 10^5$) -- počet jednotiek, ktoré treba zabiť. Tretí riadok obsahuje $n$ celých čísel $h_1,h_2,\dots,h_n$ ($1\leq h_i\leq 10^6$),
kde $h_i$ je počet životov $i$-tej jednotky, ktorú treba zabiť.


\heading{Výstup}

Pre každú jednotku vypíšte jeden riadok obsahujúci dve čísla -- $d$ a $b$, ktoré dávajú maximálnu pravdepodobnosť zabitia tejto jednotky.
Vaša odpoveď je považovaná za správnu, ak sa pravdepodobnosť zabitia jednotky pri vašom výstupe líši od optimálnej
nanajvýš o $10^{-6}$.

\heading{Podúlohy}
\begin{center}
\begin{tabular}{|c|c|c|}
	\hline
	Podúloha & Obmedzenia  & Body \\
	\hline
	1 & $m \leq 10$ & 10 \\
	\hline
	2 & $m \leq 100$ & 10 \\
	\hline
	3 & $m \leq 1\,000$ & 20 \\
	\hline
	4 & $n=1$, $m\leq 100\,000$ & 20 \\
	\hline
	5 & $m \leq 100\,000$ & 25 \\
	\hline
	6 & bez ďalších obmedzení & 15 \\
	\hline
\end{tabular}
\end{center}

\newpage
\begin{wrapfigure}{r}{4cm}
\includegraphics[width=4cm]{img/wesnoth_warning}
\end{wrapfigure}
\heading{Poznámka}
\bigskip
\bigskip
Výpočet sa dá s dostatočnou presnosťou urobiť na double-och.
Samozrejme, musíte sa vyhnúť pretečeniu, podtečeniu a nebezbečným operáciám,
ako sčítanie dvoch čísel líšiacich sa veľkosťou o veľa rádov (napr. pre $a=0.5$ a $b=10^{-100}$,
$a+b$ sa sčíta na $a$).
\bigskip
\bigskip
\bigskip
\bigskip
\heading{Príklady}

\sampleIN
10 60
3
5 6 7
\sampleOUT
5 2
2 5
7 1

\sampleEND

Zodpovedajúce pravdepodobnosti sú $84\,\%$, $68.256\,\%$, a $60\,\%$.
\end{document}
