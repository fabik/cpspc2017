\input sys/inputs.tex
\usepackage[slovak]{babel}

\begin{document}
\bigheading{Ťažký život v korporácii}

\info{corpo}{stdin}{stdout}{100}{500 ms}{1 GB}

Korporácia Sémantických Päťuholníkov je hore nohami. Nedávno bola totiž
kúpená Wall-Trumpovou Firmou a chystajú sa v nej veľké zmeny. Ale o tom neskôr,
najprv si potrebujeme vysvetliť organizačnú štruktúru firmy.

Firma má jedného prezidenta. Prezident má niekoľko podriadených vice-prezidentov.
Vice-prezidenti majú niekoľko manažérov, manažéri vicemanažérov, vicemanažéri
hypertímlídrov, hypertímlídri supertíml...

Celkovo má teda firma hlavného šéfa a veľa pracovníkov. Každý z pracovníkov okrem šéfa
má práve jedného priameho nadriadeného (ale môže mať ľubovoľný počet priamych 
podriadených). Pojmy ,,priamy podriadený'' a ,,priamy nadriadený'' sa tradičným spôsobom
dajú rozšíriť na ,,nadriadený'' a ,,podriadený'': Ak $A$ je nadriadený $B$ a $B$
je priamy nadriadený $C$, potom $A$
je nadriadeným $B$. A podobne, ak $A$ je nadriadeným $B$, potom $B$ je podriadeným $A$.

Ale naspäť k veľkým zmenám. Noví oligarchovia sa rozhodli, že firmou riadne zatrasú,
nech si nikto nemyslí, že to je len tak. Preto sa niektoré miesta zrušia,
vzniknú nové exotické miesta\footnote{Napr. referent pre boj s korupciou}, niektorí ľudia sa iba
presunú... Štruktúra firmy, aj keď ostane stromová, sa totálne
zmení. Našťastie však nikoho nevyhodia a nikoho nového neprijmú; zamestnanci ostanú
tí istí. A samotný šéf ostane na svojom starom dobrom teplučkom miestečku.

Zo zmeny ostal akurát tak veľký chaos, nadriadení sa zdráhajú dávať komukoľvek
akékoľvek rozkazy, však už na ďalší deň sa môžu ich podriadení zmeniť na
nadriadených!

\heading{Úloha}

Dostanete popis firemnej štruktúry pred zmenou a po zmene. Pre každého zamestnanca
zistite počet iných zamestnancov, ktorí boli jeho podriadenými pred zmenou a ostanú
mu podriadení aj po zmene.

\heading{Vstup}

Na prvom riadku dostanete číslo $n$ ($2 \leq n \leq 200\,000$): počet zamestnancov vo firme.
Zamestnanci sú očíslovaní číslami od $1$ do $n$; číslo $1$ patrí šéfovi.

Druhý riadok obsahuje popis firemnej štruktúry -- $n-1$ čísel $a_2, a_3, \ldots, a_n$, kde
číslo $a_i$ značí šéfa $i$-teho zamestnanca pred zmenou. Tretí riadok obsahuje tiež $n-1$
čísel, popis firemnej štruktúry po zmene v rovnakom formáte. Môžete predpokladať, že oba
popisy určujú strom zakorenený v $1$.

\heading{Výstup}

Vypíšte jeden riadok s $n$ číslami oddelenými medzerou -- $i$-te číslo označuje počet
zamestnancov, ktorí sú podriadení zamestnancovi $i$ v oboch štruktúrach.

\heading{Podúlohy}

\centering
\begin{tabular}{|l|l|l|}
\hline
Podúloha & Body & Maximálne $n$  \\ \hline
1       & 30     & 2\,000          \\ \hline
2       & 70     & 200\,000        \\ \hline
\end{tabular}

\heading{Samples}

\sampleIN
5
1 1 3 3
1 2 3 1
\sampleOUT
4 0 1 0 0 

\sampleEND

\end{document}
