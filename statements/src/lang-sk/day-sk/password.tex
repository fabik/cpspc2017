\input sys/inputs.tex
\usepackage[slovak]{babel}

\begin{document}

\bigheading{Pošahané heslo}

% \info{task_name}{infile}{outfile}{points}{timelimit}{memlimit}
% leave this values, if you are not interested
\info{password}{stdin}{stdout}{100}{1000 ms}{1 GB}

Bol nádherný deň. Slnko svietilo, voda lákala. A Alica? Práve prijímala prísne
tajné heslo k dešifrovaniu ešte prísnejšie tajných dokumentov. Poctivo si ho
celé zapísala. A potom začali heslo opakovať ešte raz od začiatku. Aj to si Alica
poctivo zapisuje, však aj malá chybička môže mať fatálne následky.

A veruže to nevyzerá dobre. Kradmé pohľady naznačujú, že v jednom znaku sa Alica
pomýlila. Ach nie! Kým tak pozerá na chybu, zabudla ďalej písať a nestihla dopísať
druhé opakovanie hesla do konca! A aby toho nebolo málo, heslo aj s opakovaním
písala hneď za seba a nepoznačila si koniec! Skončila teda s čudným reťazcom, v
ktorom sa skrýva heslo. Kto jej ho pomôže obnoviť? V stávke sú životy mnohých!

\heading{Úloha}

Dostanete reťazec $S[0 \ldots (n-1)]$ malých znakov anglickej abecedy. Nájdite najmenšiu pozíciu $i$
v reťazci, ktorou by sa mohlo začínať druhé opakovanie hesla. Formálne, nájdite $i$ spĺňajúce
nasledujúce podmienky:

\begin{enumerate}
    \item $\frac{n}{2} \leq i < n$
    \item Medzi podreťazcami $S[i \ldots (n-1)]$ a $S[0 \ldots (n - i - 1)]$ je
    rozdiel v \textbf{práve} jednom znaku.
    \item Zo všetkých hodnôt spĺňajúcich predošlé dve podmienky je $i$ tá najmenšia.
\end{enumerate}

\heading{Vstup}

Prvý riadok vstupu obsahuje číslo $n$ ($2 \leq n \leq 500\,000$), počet znakov v reťazci.
Druhý riadok vstupu obsahuje reťazec $S$ pozostávajúci z malých písmen anglickej abecedy.

\heading{Výstup}

Vypíšte jediný riadok s indexom $i$ spĺňajúcom podmienky. Ak taký index neexistuje,
vypíšte $-1$.

\heading{Podúlohy}

\centering
\begin{tabular}{|l|l|l|}
\hline
podúloha & body & maximálne $n$  \\ \hline
1       & 25     & 20           \\ \hline
2       & 25     & 5000         \\ \hline
3       & 50     & 500000       \\ \hline
\end{tabular}


\heading{Príklad}


\sampleIN
8
abaaaaaa
\sampleOUT
4
\sampleCOMMENT
Zodpovedajúce reťazce sú \texttt{abaa} a \texttt{aaaa}.
\sampleEND


\bigskip


\sampleIN
9
abcdefghi
\sampleOUT
8
\sampleEND

\bigskip

\sampleIN
5
aaaaa
\sampleOUT
-1
\sampleEND

\end{document}
