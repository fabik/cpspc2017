\input sys/inputs.tex
\usepackage[slovak]{babel}
\begin{document}

\bigheading{Zátvorky}

% \info{task_name}{infile}{outfile}{points}{timelimit}{memlimit}
% leave this values, if you are not interested
\info{brackets}{stdin}{stdout}{100}{100 ms}{1 GB}



% This is how you can include a picture:

% \includegraphics[height=5cm]{img/picture.jpg}

Paulínka išla na CPSPC a tam dostala nasledujúcu úlohu:

\heading{Úloha}

Za \emph{zátvorky} v tejto úlohe považujeme nasledujúce znaky: \texttt{()[]\{\}<>}. Korektná \emph{zátvorková postupnosť} je každý reťazec pozostávajúci zo zátvoriek, v ktorom:
\begin{itemize}
 \item Každú ľavú zátvorku zatvára nejaká pravá zátvorka rovnakého typu a každá pravá zátvorka zatvára práve jednu ľavú zátvorku.
 \item Žiadne dve dvojice prislúchajúcich zátvoriek sa "nekrižujú" -- buď sú za sebou, alebo je jedna vnútri druhej.
\end{itemize}
Napríklad, \texttt{([])<>} je korektná zátvorková posupnosť, ale \texttt{<\{>\}} nie je, keďže huňaté a zobákové zátvorky sa križujú.

Z neba nám spadol orientovaný graf s $n$ vrcholmi, v ktorom je každá hrana označená jednou zátvorkou. Sled \footnote{cesta, v ktorej sa môžu vrcholy aj hrany opakovať} v tomto grafe nazývame \emph{platný}, ak
označenia jeho hrán tvoria korektnú zátvorkovú postupnosť. Pre vrcholy $s$ a $t$ (ktoré nám tiež spadli
z neba) nájdite najkratší platný sled z $s$ do $t$.

\heading{Vstup}

Na prvom riadku vstupu sú 4 celé čísla $n, m, s, t$ ($1 \leq n \leq 200$, $0 \leq m \leq 2000$, $1 \leq s, t \leq n$) -- počet vrcholov grafu, počet hrán grafu, číslo štartovacieho vrcholu $s$ a číslo cieľového vrcholu $t$. Vrcholy sú očíslované $1, 2, \dots, n$.

Každý z nasledujúcich $m$ riadkov obsahuje dve celé čísla $x, y$ a zátvorku $b$ ($1 \leq x, y \leq n$), ktoré popisujú hranu z $x$ do $y$ označenú zátvorkou $b$. V grafe \textbf{môžu} byť slučky a násobné hrany.

\heading{Výstup}

Vypíšte jediný riadok obsahujúci jedno celé číslo -- dĺžku najkratšieho platného sledu z $s$ do $t$. Ak
takýto sled neexistuje, vypíšte $-1$. Môžete predpokladať, že ak platný sled existuje, jeho dĺžka neprekročí $10^{18}$.

\heading{Podúlohy}

\begin{center}
\begin{tabular}{|c|c|c|}
	\hline
	Podúloha & Body  & Popis \\
	\hline
	1 & 16 & $n \leq 10, m\leq 50$ \\
	\hline
	2 & 16 & $n \leq 20, m\leq 100$\\
	\hline
	3 & 16 & $n \leq 50$\\
	\hline
	4 & 16 & $n \leq 100$ \\
	\hline
	5  & 10 & $s = 1$, $t = n$, $a < b$ pre každú hranu $(a,b)$\\
	\hline
	6 & 26  & bez ďalších obmedzení \\
	\hline
\end{tabular}
\end{center}
\heading{Príklady}


\sampleIN
4 4 1 4
1 2 (
2 2 [
2 3 ]
3 4 )
\sampleOUT
4
\sampleEND
\sampleIN
5 4 1 5
1 2 <
2 3 \{
3 4 >
4 5 \}
\sampleOUT
-1

\sampleEND



\end{document}
