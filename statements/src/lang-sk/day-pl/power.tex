\input sys/inputs.tex
\usepackage[slovak]{babel}
\begin{document}

\bigheading{Elektrárne}

% \info{task_name}{infile}{outfile}{points}{timelimit}{memlimit}
% leave this values, if you are not interested
\info{power}{stdin}{stdout}{100}{3000 ms}{1 GB}

V ďalekej budúcnosti ľudstvo našlo veľkú, pohostinnú planétu a začalo ju kolonizovať. Momentálne inžinieri stavajú
hromadu elektrární, ktoré budú novej kolónii dodávať (prekvapivo) elektrinu. Vďaka skvelej novej
technológii dokážu elektrárne produkovať elektrinu z Vesmírnej Energie. 

Inžinieri však zabudli na jeden detail: ak postavíte dve elektrárne príliš blízko pri sebe, vzniká veľké riziko reťazovej reakcie, čo by v prípade havárie mohlo viesť k magnificentnej explózii. Tomuto sa, samozrejme, treba vyhnúť.

Našťastie existujú dva druhy Vesmírnej Energie, ktoré sa dajú používať v elektrárňach: \textit{ľahká}
a \textit{tmavá}. Tieto dva druhy spolu neinteragujú, takže ak susedné elektrárne používajú rôzne druhy
energie, reťazová reakcia tam nehrozí.

\heading{Úloha}

Dostanete pozície $n$ elektrární (planéta je obrovská, takže elektrárne si môžeme predstaviť ako body v rovine). Každej elektrárni určite, aký druh Energie má používať, aby Euklidovská vzdialenosť medzi
dvoma najbližšími elektrárňami rovnakého typu bola čo najväčšia.

\heading{Vstup}

V prvom riadku vstupu je jedno celé číslo $N$ ($3 \leq N \leq 100\,000$) -- počet elektrární.
Nasleduje $N$ riadkov, $i$-ty z nich obsahuje dve celé čísla $x_i$, $y_i$ ($0 \leq x_i, y_i \leq 10^9$) -- súradnice $i$-tej elektrárne. Všetky body sú rôzne.

\heading{Výstup}

Na prvý riadok vypíšte jedno celé číslo -- štvorec maximálnej dosiahnuteľnej vzdialenosti medzi najbližšími elektrárňami rovnakého typu.
Na druhý riadok vypíšte počet elektrární používajúcich ľahkú Energiu a na tretí riadok vypíšte zoznam
ich čísel, oddelených medzerami (elektrárne sú číslované od $1$).
Na štvrtý a piaty riadok vypíšte rovnakým spôsobom popis elektrární na tmavú Energiu. Ak existuje viac
riešení, vypíšte ľubovoľné z nich.

\heading{Podúlohy}

\begin{center}
\begin{tabular}{|l|l|l|}
\hline
Podúloha & Body & Maximálne $n$  \\ \hline
1       & 10     & 100           \\ \hline
2       & 25     & 2000         \\ \hline
3       & 65     & 100\,000       \\ \hline
\end{tabular}
\end{center}

\heading{Príklady}


\sampleIN
4
0 3
0 0
3 0
3 3
\sampleOUT
18
2
1 3
2
2 4
\sampleCOMMENT
\sampleEND



\end{document}
