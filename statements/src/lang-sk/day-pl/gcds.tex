\input sys/inputs.tex
\usepackage[slovak]{babel}

\begin{document}

\bigheading{Súčet spoločných deliteľov}

% \info{task_name}{infile}{outfile}{points}{timelimit}{memlimit}
% leave this values, if you are not interested
\info{gcds}{stdin}{stdout}{100}{? ms}{1 GB}

Usáma máva po večeroch flešbeky do dávnej minulosti, keď času bolo ešte dosť, najmä
na počítanie rôznych prasačiniek. Jednej z jeho príjemných spomienok na zoznamovanie
sa s teóriou čísel sa teraz budete venovať.

\heading{Úloha}

Dostanete multimnožinu $n$ prirodzených čísel. Tieto čísla budeme deliť do $k$ skupín,
v každej spočítame najväčšieho spoločného deliteľa a tieto delitele sčítame.

Pre každé $k$ od $1$ po $n$ spočítajte, aký najväčší súčet môžeme takto dostať.

\heading{Vstup}

V prvom riadku vstupu dostanete číslo $n$ ($1 \leq n \leq 500\,000$), počet čísel. V
druhom riadku dostanete $n$ prirodzených čísel $a_1, \ldots, a_n$ oddelených medzerami.
Platí $1 \leq a_i \leq 10^{12}$.

\heading{Výstup}

Vypíšte $n$ riadkov, každý obsahujúc jedno číslo -- najväčšie možné súčty
najmenších spoločných deliteľov pri delení do $1$, $2$, \ldots, $n$ skupín (v tomto poradí).

\heading{Podúlohy}

\begin{center}
  \begin{tabular}{|c|p{8cm}|c|}
    \hline
    \textbf{Podúloha} & \textbf{Obmedzenia} & \textbf{Body} \\ \hline
    1 & $n \leq 7$ & 5 \\ \hline
    2 & $n \leq 15$ & 5 \\ \hline
    3 & $n \leq 100$, $a_i \leq 500$ & 8 \\ \hline
    4 & $n \leq 2000$, $a_i \leq 2000$, hodnoty $a_i$ sú rôzne & 8 \\ \hline
    5 & $n \leq 2000$ & 14 \\ \hline
    6 & Hodnoty $a_i$ sú rôzne & 25 \\ \hline
    7 & Bez špeciálnych obmedzení & 35 \\ \hline
  \end{tabular}
\end{center}

\heading{Príklad}
\sampleIN
4
10 9 10 3
\sampleOUT
1
13
23
32

\sampleEND

\noindent \textit{Pre $k = 2$, najlepšie rozdelenie je do skupín $(10,10)$ a $(9,3)$ so súčtom $10+3 = 13$. Pre $k=3$, najlepšie rozdelenie je $(10)$, $(10)$ a $(9,3)$ so súčtom $23$.}

\medskip

\sampleIN
8
15 25 29 30 43 44 45 55
\sampleOUT
1
56
101
145
188
221
256
286

\sampleEND



\end{document}
