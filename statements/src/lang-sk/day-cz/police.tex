\input sys/inputs.tex
\usepackage[slovak]{babel}

\begin{document}

\bigheading{Policajná stanica}

% \info{task_name}{infile}{outfile}{points}{timelimit}{memlimit}
% leave this values, if you are not interested
\info{police}{stdin}{stdout}{100}{2000 ms}{1 GB}

Po nedávnom fiasku a následných protestoch sa polícia rozhodla, že vybuduje
hyperrýchle hyperdiaľnice. Tak sa budú môcť policajti dostať na každé miesto takmer
okamžite a budú vedieť predísť tragickým udalostiam.

Vyhlásili sa rýchle tendre, vybudovali sa hyperdiaľnice pochybnej kvality. A to
najhoršie: Vybudovali sa jednosmerné a úplne bez ladu a skladu.

Vrámci šetrenia chce vláda postaviť jedinú policajnú základňu so špeciálnym
vybavením pre cestovanie po hyperdiaľniciach. Z tejto základne musí existovať
cesta po hyperdiaľniciach do každého mesta.

Spätná cesta po hyperdiaľniciach nemusí existovať. Policajti sa po úspešnom
zásahu môžu vrátiť na stanicu aj po pomalých medzimestských cestách.

Vládni poradcovia a experti si momentálne lámu hlavu, ale ani za svet nevedia
nájsť mestá, v ktorých by sa potenciálne mohla základňa vybudovať. Preto požiadali
o pomoc práve vás!

\heading{Úloha}

Máte zadanú sieť jednosmerných hyperdiaľnic. Nájdite všetky mestá, z ktorých
sa dá dostať do každého iného mesta iba nejakou postupnosťou hyperdiaľnic.

\heading{Vstup}

Prvý riadok vstupu obsahuje dve celé čísla $N$ a $M$: počet miest a počet
hyperdiaľnic.

Ďalej nasleduje $M$ riadkov. Každý z nich obsahuje dve celé čísla $A_i$ a $B_i$:
čísla miest, ktoré spája $i$-ta hyperdiaľnica. Každá hyperdiaľnica je jednosmerná a vedie
z mesta $A_i$ do mesta $B_i$.

\bigskip
\noindent
Platí, že $1 \leq N, M \leq 10^6$.\
Pre každé $i$ platí, že $1 \leq A_i, B_i \leq N$ a $A_i \neq B_i$.\\
Žiadne dve hyperdiaľnice nespájajú tie isté mestá tým istým smerom. Môžu však
spájať dve mestá v opačných smeroch. \\
V 30 \% vstupov navyše platí $N \leq 10^3$ a $M \leq 3 \cdot 10^3$.

\heading{Výstup}

Vypíšte práve dva riadky. V prvom riadku má byť celkový počet miest, z ktorých
sa dá dostať do každého iného mesta. V druhom riadku majú byť medzerami oddelené
čísla týchto miest. \textbf{Čísla miest musia byť usporiadané vzostupne!}

\heading{Príklad}

\sampleIN
5 6
1 3
1 4
4 2
2 1
2 5
5 4
\sampleOUT
4
1 2 4 5
\sampleEND

\sampleIN
3 2
1 3
2 3
\sampleOUT
0
~
\sampleEND

\end{document}
