\input sys/inputs.tex
\usepackage[slovak]{babel}

\begin{document}

\bigheading{Robotie preteky}

% \info{task_name}{infile}{outfile}{points}{timelimit}{memlimit}
% leave this values, if you are not interested
\info{race}{stdin}{stdout}{100}{2000 ms}{1 GB}

V Absurdistane sa chystá ďalší ročník populárnych robotích pretekov. Tento rok
si organizátori prichystali novinku: roboty sa budú pretekať v bludisku. Bludisko
má tvar obdĺžnika a je rozdelené na $n$ riadkov po $m$ štvorčekov.
Každý štvorček je buď prázdny, alebo je v ňom prekážka.

%A robot race on a maze will be held in Byteland. The maze has a shape of
%a rectangle and is divided into $n \times m$ fields arranged in $n$ rows and $m$
%columns. Each of the fields is either empty or contains an obstacle.

Súťažiaci chodia k bludisku po jednom. Každý súťažiaci si vylosuje súradnice
štartovacieho a cieľového políčka. Jeho robota potom postavia na štartovacie políčko
a zapnú. Cieľom robota je čo najrýchlejšie sa dostať na cieľové políčko.

%The contestants register their robots for the competition and come to the maze
%one after another. Each of the contestants gets random coordinates of the
%initial field and the target field. The robot is then placed in the initial
%field and must get to the target field using a sequence of steps.

Aby to nebola taká nuda,
pridali organizátori ešte jedno pravidlo: v každom kroku sa robot môže pohnúť iba o jedno
políčko smerom na juh (na mape dodola) alebo na východ (doprava). Samozrejme, nemôže tým 
vstúpiť na políčko s prekážkou.

%To make the game more challenging, the rules specify that in each step the robot
%can only move one field right or one field down in the maze. Moving the robot in
%any other direction is not allowed.

%The contestant whose robot gets fastest from the initial field to the target
%field wins the competition. If the robot does not get to the target field in the
%time limit, the contestant in disqualified.

Pri takýchto pravidlách sa môže stať, že súťažiaci si vylosuje zlú dvojicu políčok
a jeho robot sa nemá šancu dostať zo štartovacieho políčka do cieľového. V takom prípade
chcú organizátori nechať súťažiaceho losovať znovu. Najprv však musia zistiť, že táto
situácia nastala.

%The organizers of the competition realized that if the contestant gets bad
%coordinates, the robot will not be able to get to the target field using any
%sequence of moves. In that case they would like to give the contestant another
%pair of coordinates.

\heading{Úloha}

Dostanete mapu bludiska $n \times m$ a $q$ dvojíc (štartovacie políčko, cieľové políčko).
Pre každú takúto dvojicu určite, či sa zo štartovacieho políčka dá dostať do cieľového
iba pomocou krokov na juh a na východ.

%You are given a map of a maze of size $n \times m$ and $q$ pairs of coordinates
%of the initial and target fields. Determine for each pair of coordinates whether
%it is possible to get from the initial field to the target field using
%a sequence of steps right or down.

\heading{Vstup}

Prvý riadok vstupu obsahuje tri celé čísla $n, m$ a $q$: počet riadkov bludiska,
počet stĺpcov bludiska a počet dvojíc (štartovacie políčko, cieľové políčko).

Každý z nasledujúcich $n$ riadkov obsahuje $m$ znakov popisujúcich jednotlivé políčka
bludiska. Znak \texttt{.} reprezentuje prázdne políčko a znak \texttt{\#} reprezentuje
políčko s prekážkou.

%The first line contains three integers $n$, $m$ and $q$: the number of rows and
%columns of the maze and the number of pairs of coordinates.

%Each of the following $n$ lines contains $m$ characters describing the fields of
%the maze. The character \texttt{.} represents an empty field and the character
%\texttt{\#} a field with an obstacle.

Nasleduje $q$ riadkov s dvojicami políčok, $i$-ty z nich obsahuje štyri celé čísla
${r_1}_i, {c_1}_i, {r_2}_i, {c_2}_i$: riadok a stĺpec štartovacieho políčka a riadok
a stĺpec cieľového políčka.

%Then $q$ lines follow describing the pairs of coordinates. The $i$-th of them
%contains four integers ${r_1}_i$, ${c_1}_i$, ${r_2}_i$, ${c_2}_i$: the row and
%the column of the initial field and the row and the column of the target field.

\bigskip
Platia nasledovné obmedzenia:
\begin{itemize}

\item $1 \leq n, m \leq 1\,000$
\item $1 \leq q \leq 10^6$
\item Pre všetky $i \in \{ 1, 2, \ldots, q \}$ platí $1 \leq {r_1}_i, {r_2}_i \leq n$
      a $1 \leq {c_1}_i, {c_2}_i \leq m$.
\item Pre všetky $i \in \{ 1, 2, \ldots, q \}$ sú políčka so súradnicami
        $({r_1}_i, {c_1}_i)$ a $({r_2}_i, {c_2}_i)$ voľné.
\item Navyše, vo vstupoch za 20 \% bodov platí $q \leq 300$.
\end{itemize}

\heading{Výstup}

Vypíšte $q$ riadkov. Ak sa z $i$-teho štartovacieho políčka dá dostať na $i$-te cieľové políčko,
$i$-ty riadok má obsahovať slovo \texttt{YES}. V opačnom prípade má $i$-ty riadok byť \texttt{NO}.

\heading{Príklady}

\sampleIN
3 4 5
.\#..
.\#\#.
....
1 1 3 4
1 3 3 4
1 1 1 1
1 1 2 4
2 1 2 4
\sampleOUT
YES
YES
YES
NO
NO
\sampleEND

\end{document}
