\input sys/inputs.tex

\begin{document}

\bigheading{Těžký život v korporaci}

% \info{task_name}{infile}{outfile}{points}{timelimit}{memlimit}
% leave this values, if you are not interested
\info{corpo}{stdin}{stdout}{100}{500 ms}{1 GB}

Korporace sémantických pětiúhelníků je vzhůru nohama. Nedávno byla totiž koupená
Wall-Trumpovou firmou a chystají se v ní velké změny. Ale o tom později, nejprve
si potřebujeme vysvětlit organizační strukturu firmy.

Firma má jednoho prezidenta. Prezident má několik podřízených vice-prezidentů.
Vice-prezidenti mají několik podřízených manažerů, manažeři vicemanažerů,
vicemanažeři hypertýmlídrů, hypertýmlídři supertýml...

Celkově má tedy firma hlavního šéfa a mnoho pracovníků. Každý z pracovníků kromě
šéfa má právě jednoho přímého nadřízeného (ale může mít libovolný počet přímých
podřízených). Pojmy \uv{přímý podřízený} a \uv{přímý nadřízený} se tradičním
způsobem dají rozšířit na \uv{nadřízený} a \uv{podřízený}: Pokud je $A$
nadřízeným $B$ a $B$ je přímým nadřízeným $C$, pak je $A$ nadřízeným $C$. Obdobně,
pokud je $A$ nadřízeným $B$, potom je $B$ podřízeným $A$.

Ale zpět k velkým změnám. Noví oligarchové se rozhodli, že firmou řádně zatřesou,
ať si nikdo nemyslí, že je to jen tak. Proto se některá místa zruší, vzniknou
nová exotická místa\footnote{např. referent pro boj s korupcí}, někteří lidé se
pouze přesunou... Struktura firmy, i když zůstane stromová, se totálně změní. A
samotný šéf zůstane na svém starém dobrém teploučkém místečku.

Ze změny zůstal akorát tak velký chaos, nadřízení se zdráhají dávat komukoliv
jakékoli rozkazy, neboť se hned následující den mohou jejich podřízení stát
jejich nadřízenými!

\heading{Úloha}

Dostanete popis firemní struktury před změnou a po změně. Pro každého zaměstnance
zjistěte počet jiných zaměstnanců, kteří byli jeho podřízenými před změnou a
zůstanou mu podřízení i po změně.

\heading{Vstup}

Na prvním řádku dostanete číslo $n$ ($2 \leq n \leq 200\,000$): počet zaměstnanců ve firmě.
Zaměstnanci jsou očíslovaní čísly od $1$ do $n$; číslo $1$ patří šéfovi.

Druhý řádek obsahuje popis firemní struktury -- $n-1$ čísel $a_2, a_3, \ldots, a_n$, kde
číslo $a_i$ značí šéfa $i$-tého zaměstnance před změnou. Třetí řádek obsahuje také $n-1$
čísel, popis firemní struktury po změně ve stejném formátu. Můžete předpokládat, že oba
popisy určují strom zakořeněný v $1$.

\heading{Výstup}

Vypište jeden řádek s $n$ čísly oddělenými mezerou -- $i$-té číslo označuje počet
zaměstnanců, kteří jsou podřízení zaměstnanci $i$ v oboch strukturách.

\heading{Podproblémy}

\centering
\begin{tabular}{|l|l|l|}
\hline
podproblém & body & nejvyšší $n$  \\ \hline
1       & 30     & 2\,000          \\ \hline
2       & 70     & 200\,000        \\ \hline
\end{tabular}

\heading{Příklad}

\sampleIN
5
1 1 3 3
1 2 3 1
\sampleOUT
4 0 1 0 0 
\sampleEND

\end{document}
