\input sys/inputs.tex

\begin{document}

\bigheading{Bitva o Wesnoth}

% \info{task_name}{infile}{outfile}{points}{timelimit}{memlimit}
% leave this values, if you are not interested
\info{wesnoth}{stdin}{stdout}{100}{100 ms}{1 GB}

Znáte Bitvu o Wesnoth? Je to vynikající tahová strategie na motivy fantasy.
Ve hře potkáte spoustu elfů, skřetů, nemrtvých, trpaslíků i zuřících draků!
Co se týče naší úlohy, důležité je, že každá jednotka má nějaký počet životů,
které jí během soubojů ubývají. Dále musíte pochopit, jak takový souboj probíhá.
Pro naše účely stačí vysvětlit ten nejjednodušší příklad --
běžný útok na jednotku, která nemá žádnou obranu.

Takový útok lze popsat pomocí tří celých čísel:
\begin{itemize}
 \item $d$ -- poškození způsobené úspěšným úderem;
 \item $b$ -- počet úderů;
 \item $p$ -- pravděpodobnost (v procentech), že úder bude úspěšný.
\end{itemize}
Ve hře jsou $d$ a $b$ vlastnosti útočníka, zatímco $p$ je obvykle určeno podle terénu,
na kterém stojí obránce; nicméně pro naše účely budeme i $p$ považovat za vlastnost útočníka
(jako je tomu u magických útoků).

Jako příklad uvažme útok s $d=6$, $b=2$ a $p=60$. Jsou tři možnosti, jak může tento útok dopadnout:
\begin{itemize}
 \item S pravděpodobností $16\,\%$ útočník mine v obou případech a nezpůsobí žádné zranění.
 \item S pravděpodobností $48\,\%$ se útočník trefí jedním z úderů a mine druhým, čímž způsobí zranění $6$ životů.
 \item S pravděpodobností $36\,\%$ se útočník trefí v obou případech, čímž dohromady způsobí zranění $12$ životů.
\end{itemize}
Jakmile počet životů bránící jednotky klesne na nekladné číslo, jednotka zemře.

David právě hraje rozšíření, kde je speciální jednotka elfí princezna, která má zvláštní magický útok:
Místo čísel $d$ a $b$ je popsána celým číslem $m$, a kdykoli útočí, může si hráč zvolit
za $d$ a $b$ libovolná kladná celá čísla splňující $d\cdot b\leq m$.

Tu a tam David potřebuje, aby jeho elfí princezna zabila ošklivého nemrtvého kostlivce nebo páchnoucího skřetího válečníka,
a tak by David rád věděl, která volba čísel $d$ a $b$ povede k největší pravděpodobnosti, že nepřátelská jednotka zemře.

\heading{Úloha}

Pro každou jednotku, kterou David potřebuje zabít, zjistěte nejlepší výběr čísel $d$ a $b$.

\heading{Vstup}

První řádek vstupu obsahuje celá čísla $m$ ($1\leq m\leq 10^6$)
a $p$ ($1\leq p\leq 99$) -- parametry útočící jednotky.
Druhý řádek obsahuje celé číslo $n$ ($1\leq n\leq 10^5$) -- počet jednotek na zabití.
Třetí řádek obsahuje $n$ celých čísel $h_1,h_2,\dots,h_n$ ($1\leq h_i\leq 10^6$),
kde $h_i$ je počet životů $i$-té jednotky na zabití.


\heading{Výstup}

Pro každou jednotku na zabití vypište dvě čísla: hodnoty $d$ a $b$, pro které je nejvyšší možná
šance, že nepřátelská jednotka souboj nepřežije. Vaše odpověď bude uznána, pokud se pravděpodobnost
při vašem návrhu bude od optimální hodnoty lišit o nejvýš $10^{-6}$.

\heading{Podproblémy}

\begin{tabular}{|c|c|c|}
	\hline
	Podproblém & Omezení  & Body \\
	\hline
	1 & $m \leq 10$ & 10 \\
	\hline
	2 & $m \leq 100$ & 10 \\
	\hline
	3 & $m \leq 1\,000$ & 20 \\
	\hline
	4 & $n=1$, $m\leq 100\,000$ & 20 \\
	\hline
	5 & $m \leq 100\,000$ & 25 \\
	\hline
	6 & žádná omezení & 15 \\
	\hline
\end{tabular}

\heading{Poznámka}

Celý výpočet lze provést s více než dostatečnou přesností pomocí doublů.
Samozřejmě se musíte vyvarovat přetečení, podtečení a provádění nebezpečných operací
jako sčítání čísel velmi odlišných velikostí (např. pro $a=0.5$ a $b=10^{-100}$
by $a+b$ bylo vyhodnoceno jako rovno $a$).
\heading{Příklad}

\sampleIN
10 60
3
5 6 7
\sampleOUT
5 2
2 5
7 1

\sampleEND

Odpovídající pravděpodobnosti jsou $84\,\%$, $68.256\,\%$ a $60\,\%$.
\end{document}
