\input sys/inputs.tex
\usepackage[czech]{babel}

\begin{document}

\bigheading{Policejní stanice}

% \info{task_name}{infile}{outfile}{points}{timelimit}{memlimit}
% leave this values, if you are not interested
\info{police}{stdin}{stdout}{100}{2000 ms}{1 GB}

V galaxii je vybudována síť hyperprostorových dálnic. Každá dálnice tvoří
jednosměrný tunel mezi dvěma planetami. Galaktická vláda vybírá planetu, na
které postaví policejní stanici.

Aby mohla policie efektivně chránit celou galaxii, musí být z policejní stanice
možné dopravit se prostřednictvím sítě hyperprostorových dálnic na každou
planetu v galaxii.

Zpáteční cesta z každé planety na policejní stanici po hyperprostorových
dálnicích už existovat nemusí. Zpátky už policie nepotřebuje tolik spěchat,
takže se může vydat i po pomalejších cestách než po dálnicích.

Vláda si neví rady, jak najít vhodnou planetu, na které policejní stanici
postavit, a proto požádala o pomoc vás.

\heading{Úloha}

Je zadána síť jednosměrných hyperprostorových dálnic. Najděte všechny planety,
ze kterých se dá dostat na všechny ostatní planety v galaxii pomocí nějaké
posloupnosti dálnic.

\heading{Vstup}

První řádek vstupu obsahuje dvě celá čísla $N$ a $M$: počet planet a počet
dálnic.

Poté následuje $M$ řádků. Každý z nich obsahuje dvě celá čísla $A_i$ a $B_i$:
čísla planet, které propojuje $i$-tá dálnice. Každá dálnice je jednosměrná a
vede z planety $A_i$ do planety $B_i$.

\bigskip
\noindent
Platí, že $1 \leq N, M \leq 10^6$.\
Pro každé $i$ platí, že $1 \leq A_i, B_i \leq N$ a $A_i \neq B_i$.\\
Žádné dvě dálnice nespojují stejné planety ve stejném směru. Mohou však spojovat
  dvě stejné planety v opačných směrech.\\
Ve 30 \% vstupů navíc platí $N \leq 10^3$ a $M \leq 3 \cdot 10^3$.

\heading{Výstup}

Na výstupu musí být právě dva řádky. Na prvním řádku musí být celkový počet
planet, ze kterých se dá dostat na všechny ostatní planety v galaxii. Na druhém
řádku musí být mezerou oddělená čísla těchto planet. \textbf{Čísla planet musí
být uspořádána vzestupně!}

\heading{Příklady}

\sampleIN
5 6
1 3
1 4
4 2
2 1
2 5
5 4
\sampleOUT
4
1 2 4 5
\sampleEND

\sampleIN
3 2
1 3
2 3
\sampleOUT
0
~
\sampleEND

\end{document}
