\input sys/inputs.tex

\begin{document}

\bigheading{Pyramida}

% \info{task_name}{infile}{outfile}{points}{timelimit}{memlimit}
% leave this values, if you are not interested
\info{pyramid}{stdin}{stdout}{100}{2000 ms}{1 GB}

Archeologové právě rozluštili hieroglyfy na stěnách pyramidy. Nápis na jedné ze
stěn popisuje $N$ posvátných čísel. Všechna čísla, která jsou dělitelná alespoň
jedním z těchto čísel, jsou také posvátná.

Na $M$ dalších stěnách se píše, že $Q_i$-té nejmenší posvátné číslo má magické
vlastnosti. Archeology by velmi zajímalo, která čísla mají magické vlastnosti.
Pomůžete jim to zjistit?

\heading{Úloha}

Je zadáno $N$ kladných celých čísel $A_1, A_2, \ldots, A_N$ a $M$ kladných
celých čísel $Q_1, Q_2, \ldots, Q_M$. Pro každé $i \in \{ 1, 2, \ldots, M \}$
najděte $Q_i$-té nejmenší kladné celé číslo, které má tu vlastnost, že je
dělitelné alespoň jedním z čísel $A_1, A_2, \ldots, A_N$.

\heading{Vstup}

První řádek vstupu obsahuje dvě celá čísla $N$ a $M$. Druhý řádek obsahuje
mezerou oddělená čísla $A_1$, $A_2$, $\ldots$, $A_N$. Poté následuje $M$ řádků.
Každý z nich obsahuje celé číslo $Q_i$.

\bigskip
\noindent
Platí, že $1 \leq N \leq 15$ a $1 \leq M \leq 50$.\\
Pro každé $i \in \{ 1, 2, \ldots, N \}$ platí, že $2 \leq A_i \leq 10^{18}$.\\
Pro součin těchto čísel platí, že $A_1 \cdot A_2 \cdot \ldots \cdot A_N \leq 10^{18}$.\\
Pro každé $i \in \{ 1, 2, \ldots, M \}$ platí, že $1 \leq Q_i \leq 10^{18}$.\\
Každé číslo na výstupu bude menší nebo rovno $10^{18}$.\\
V 10 \% vstupů navíc platí, že $Q_1, Q_2, \ldots, Q_M \leq 10^6$.\\
Ve 30 \% vstupů navíc platí, že $N \leq 2$.

\heading{Výstup}

Na výstupu musí být právě $M$ řádků. Na $i$-tém z nich musí být $Q_i$-té
nejmenší kladné celé číslo, které má tu vlastnost, že je dělitelné alespoň
jedním z čísel $A_1, A_2, \ldots, A_N$.

\heading{Příklad}

\sampleIN
5 5
2 5 7 10 11
1
2
3
10
20
\sampleOUT
2
4
5
14
28
\sampleEND

\sampleIN
2 1
70 100
5
\sampleOUT
210
\sampleEND

\end{document}
