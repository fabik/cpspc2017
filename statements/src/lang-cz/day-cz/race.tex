\input sys/inputs.tex
\usepackage[czech]{babel}

\begin{document}

\bigheading{Závody robotů}

% \info{task_name}{infile}{outfile}{points}{timelimit}{memlimit}
% leave this values, if you are not interested
\info{race}{stdin}{stdout}{100}{2000 ms}{1 GB}

V Bytelandii se budou konat závody robotů na bludišti. Půdorys bludiště má tvar
obdélníku a je rozdělen na $n \times m$ polí uspořádaných do $n$ řádků a $m$
sloupců. Každé z polí je buď prázdné, nebo je na něm překážka.

Soutěžící do soutěže přihlásí své roboty a poté jeden do druhém přistupují
k bludišti. Zde si vylosují souřadnice počátečního a cílového pole, umístí svého
robota na počáteční pole a robot se musí posloupností kroků dostat na cílové
pole.

Aby to ale neměli tak jednoduché, tak se podle pravidel může robot v každém
kroku v bludišti posunout buď o jedno pole dolů nebo o jedno pole doprava.
Jinými směry se pohybovat nesmí.

Vítězem se stane ten soutěžící, jehož robot se dostane z počátečního pole na
cílové nejrychleji. Pokud se robot nedokáže ve stanoveném časovém limitu dostat
do cíle, tak je soutěžící diskvalifikován.

Organizátoři si uvědomili, že když si soutěžící vylosuje špatné souřadnice, tak
se jeho robot do cíle nemůže žádnou posloupností kroků dostat. V takovém případě
by mu chtěli dát možnost vylosovat si nové souřadnice.

\heading{Úloha}

Je dána mapa bludiště o velikosti $n \times m$ a $q$ vylosovaných souřadnic
počátečního a cílového pole. Zjistěte pro každou dvojici souřadnic, zda se dá
z počátečního pole v bludišti dostat do cílového pole nějakou posloupností kroků
dolů nebo doprava.

\heading{Vstup}

První řádek vstupu obsahuje celá čísla $n$, $m$ a $q$: počet řádků a sloupců
bludiště a počet dvojic souřadnic.

Na každém z následujících $n$ řádků je $m$ znaků, které popisují jednotlivá pole
bludiště. Znak \texttt{.} značí prázdné pole, znak \texttt{\#} značí překážku.

Poté následuje $q$ řádků popisujících vylosované souřadnice. Na $i$-tém z nich
jsou čtyři celá čísla ${r_1}_i$, ${c_1}_i$, ${r_2}_i$, ${c_2}_i$: řádek a
sloupec počátečního pole a řádek a sloupec cílového pole.

\bigskip
\noindent
Platí, že $1 \leq n,m \leq 1000$ a $1 \leq q \leq 10^6$.\\
Pro každé $i \in \{ 1, 2, \ldots, q \}$ platí, že
  $1 \leq {r_1}_i, {r_2}_i \leq n$ a $1 \leq {c_1}_i, {c_2}_i \leq m$.\\
Pro každé $i \in \{ 1, 2, \ldots, q \}$ jsou pole v bludišti na souřadnicích
  $({r_1}_i, {c_1}_i)$ a $({r_2}_i, {c_2}_i)$ prázdná.\\
Ve 20 \% vstupů navíc platí, že $q \leq 300$.

\heading{Výstup}

Vypište $q$ řádků. Na $i$-tém z nich vypište slovo \texttt{YES}, pokud se robot
může z $i$-tého počátečního pole dostat na $i$-té cílové pole, nebo slovo
\texttt{NO}, pokud se na cílové pole dostat nemůže.

\heading{Příklad}

\sampleIN
3 4 5
.\#..
.\#\#.
....
1 1 3 4
1 3 3 4
1 1 1 1
1 1 2 4
2 1 2 4
\sampleOUT
YES
YES
YES
NO
NO
\sampleEND

\end{document}
