\input sys/inputs.tex

\begin{document}

\bigheading{Dláždění náměstí}

% \info{task_name}{infile}{outfile}{points}{timelimit}{memlimit}
% leave this values, if you are not interested
\info{paving}{stdin}{stdout}{100}{2000 ms}{1 GB}

V Kocourkově se bude rekonstruovat historické náměstí s mozaikovou dlažbou.
Náměstí je pokryté bílými a šedými dlažebními kostkami, které jsou uspořádané v obdélníkové mřížce.

Památkáři mají na rekonstrukci dva požadavky:
1. Každý čtverec 2x2 musí obsahovat lichý počet šedých dlaždic.
2. Některé dlažební kostky jsou památkově chráněné, takže musí zůstat na svém místě.

Radní by zajímalo, kolika způsoby je možné náměstí vydláždit tak, aby vyhověli oběma podmínkám památkářů.

\heading{Úloha}

Jsou zadány rozměry náměstí a seznam památkově chráněných dlaždic.

Spočítejte, kolika způsoby je možné zvolit barvy ostatních dlaždic tak, aby v každém čtverci 2x2 byl lichý počet šedých dlaždic.

\heading{Vstup}

První řádek vstupu obsahuje tři celá čísla $N$, $M$ a $K$: počet řádků mřížky, počet sloupců mřížky a počet chráněných dlaždic.

Poté následuje $K$ řádků. Každý z nich obsahuje tři celá čísla $R_i$, $C_i$ a $G_i$: číslo řádku a číslo sloupce, ve kterém se nachází $i$-tá dlaždice, a barvu $i$-té dlaždice (číslo 0 značí bílou barvu, číslo 1 šedou barvu).

\bigskip
\noindent
Platí, že $2 \leq N,M \leq 10^5$ a $0 \leq K \leq 10^5$.\\
Pro každé $i$ platí, že $1 \leq R_i \leq N$, $1 \leq C_i \leq M$ a $G_i \in \{ 0, 1 \}$.\\
Žádné dvě dlaždice nemají shodné souřadnice.\\
Ve $20\%$ vstupů navíc platí, že $N,M \leq 5$ a $K \leq 5$.\\
Ve $40\%$ vstupů navíc platí, že $N,M \leq 5000$ a $K \leq 25$.

\heading{Výstup}

Na výstupu musí být právě jeden řádek obsahující počet způsobů, kterými lze náměstí vydláždit.

Protože výsledné číslo může být velmi velké, vypište pouze jeho zbytek po dělení číslem $10^9$.

\heading{Příklad}

\sampleIN
3 4 3
2 2 1
1 2 0
2 3 1
\sampleOUT
8
\sampleEND

\end{document}
