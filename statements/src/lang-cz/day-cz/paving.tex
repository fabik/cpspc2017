\input sys/inputs.tex

\begin{document}

\bigheading{Dláždění náměstí}

% \info{task_name}{infile}{outfile}{points}{timelimit}{memlimit}
% leave this values, if you are not interested
\info{paving}{stdin}{stdout}{100}{2000 ms}{1 GB}

V Kocourkově se bude rekonstruovat historické náměstí s mozaikovou dlažbou.
Náměstí je pokryté bílými a šedými dlažebními kostkami, které jsou uspořádané
v obdélníkové mřížce. Památkáři mají na rekonstrukci dva požadavky:

\begin{enumerate}[nolistsep]
\item Každý čtverec 2x2 musí obsahovat lichý počet šedých dlaždic.
\item Některé dlažební kostky jsou památkově chráněné, takže musí zůstat na svém
  místě.
\end{enumerate}

Radní by zajímalo, kolika způsoby je možné náměstí vydláždit tak, aby vyhověli
oběma podmínkám památkářů. Pomůžete jim?

\heading{Úloha}

Jsou zadány rozměry náměstí a seznam památkově chráněných dlaždic. Spočítejte,
kolika způsoby je možné zvolit barvy ostatních dlaždic tak, aby v každém čtverci
2x2 byl lichý počet šedých dlaždic.

\heading{Vstup}

První řádek vstupu obsahuje tři celá čísla $N$, $M$ a $K$: počet řádků mřížky,
počet sloupců mřížky a počet památkově chráněných dlaždic.

Poté následuje $K$ řádků popisujících jednotlivé památkově chráněné dlaždice.
Každý z nich obsahuje tři celá čísla $R_i$, $C_i$ a $G_i$: číslo řádku a číslo
sloupce, ve kterém se nachází $i$-tá dlaždice, a její barvu (číslo 0 značí bílou
barvu, číslo 1 šedou barvu).

\bigskip
\noindent
Platí, že $2 \leq N,M \leq 10^5$ a $0 \leq K \leq 10^5$.\\
Pro každé $i$ platí, že $1 \leq R_i \leq N$, $1 \leq C_i \leq M$ a
  $G_i \in \{ 0, 1 \}$.\\
Žádné dvě památkově chráněné dlaždice nemají shodné souřadnice.\\
Ve 20 \% vstupů navíc platí, že $N,M \leq 5$ a $K \leq 5$.\\
Ve 40 \% vstupů navíc platí, že $N,M \leq 5000$ a $K \leq 25$.

\heading{Výstup}

Na výstupu musí být právě jeden řádek obsahující počet způsobů, kterými lze
náměstí vydláždit. Protože výsledné číslo může být velmi velké, vypište pouze
jeho zbytek po dělení číslem $10^9 + 7$.

\heading{Příklad}

\sampleIN
3 4 3
2 2 1
1 2 0
2 3 1
\sampleOUT
8
\sampleEND

\end{document}
