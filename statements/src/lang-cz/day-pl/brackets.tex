\input sys/inputs.tex
\usepackage[czech]{babel}

\begin{document}

\bigheading{Závorky}

% \info{task_name}{infile}{outfile}{points}{timelimit}{memlimit}
% leave this values, if you are not interested
\info{brackets}{stdin}{stdout}{100}{200 ms}{1 GB}



% This is how you can include a picture:

% \includegraphics[height=5cm]{img/picture.jpg}

\heading{Úloha}

Závorkou rozumíme jeden z následujících symbolů: \texttt{()[]\{\}<>}. Řetězec závorek nazveme \textit{dobře uzávorkovaný},
právě když existuje vzájemně jednoznačné zobrazení mezi levými a pravými závorkami takové, že:
\begin{itemize}
 \item Každá levá závorka je nalevo od odpovídající pravé závorky.
 \item Odpovídající si závorky jsou vždy stejného typu.
 \item Žádné dva páry odpovídajících si závorek se nekříží -- buď je jeden celý nalevo od druhého, nebo je jeden celý uvnitř druhého.
\end{itemize}
Například výraz \texttt{([])<>} je dobře uzávorkovaný, zatímco \texttt{<\{>\}} není, jelikož páry odpovídajících si závorek se kříží.

Je dán orientovaný multigraf, kde každá hrana je označena nějakou závorkou.
Sled (tj. cestu, kde se můžou libovolně opakovat vrcholy i hrany)
nazveme \textit{validní}, právě když výraz přečtený po jeho hranách je dobře uzávorkovaný.
Vaším úkolem je zjistit délku nejkratšího validního sledu mezi danými dvěma vrcholy $s$ a $t$.


\heading{Vstup}

Na prvním řádku vstupu jsou celá čísla $n, m, s, t$ ($1 \leq n \leq 200$, $0 \leq m \leq 2000$, $1 \leq s, t \leq n$) -- počet vrcholů, počet hran, číslo startovního vrcholu a číslo cílového vrcholu.
Následuje $m$ řádků, které popisují jednotlivé hrany. Každý řádek obsahuje počáteční vrchol $x$, koncový vrchol $y$
($1 \leq x, y \leq n$) a závorku. Připomínáme, že se jedná o multigraf, tedy může obsahovat smyčky ($x=y$)
a násobné hrany.

\heading{Výstup}

Na jediný řádek výstupu vypište jediné číslo -- délku nejkratšího validního sledu mezi $s$ a $t$.
Pokud žádný neexistuje, vypište $-1$. Můžete předpokládat, že správná odpověď nikdy nepřesáhne $10^{18}$.

\heading{Podproblémy}

\begin{tabular}{|c|c|c|}
	\hline
	Podproblém & Body  & Omezení \\
	\hline
	1 & 16 & $n \leq 10, m\leq 50$ \\
	\hline
	2 & 16 & $n \leq 20, m\leq 100$\\
	\hline
	3 & 16 & $n \leq 50$\\
	\hline
	4 & 16 & $n \leq 100$ \\
	\hline
	5  & 10 & $s = 1$, $t = n$, $a < b$ pro každou hranu $(a,b)$\\
	\hline
	6 & 26  & žádná omezení \\
	\hline
\end{tabular}

\heading{Příklady}


\sampleIN
4 4 1 4
1 2 (
2 2 [
2 3 ]
3 4 )
\sampleOUT
4
\sampleEND
\sampleIN
5 4 1 5
1 2 <
2 3 \{
3 4 >
4 5 \}
\sampleOUT
-1

\sampleEND



\end{document}
