\input sys/inputs.tex
\usepackage[czech]{babel}

\begin{document}

\bigheading{Elektrárny}

% \info{task_name}{infile}{outfile}{points}{timelimit}{memlimit}
% leave this values, if you are not interested
\info{power}{stdin}{stdout}{100}{3000 ms}{1 GB}

Ocitáme se v daleké budoucnosti, kdy se lidem podařil opravdový majstrštych. Našli a kolonizovali nedalekou planetu,
která by mohla být skvělým zdrojem energie, a proto by nyní bylo vhodné zde postavit několik elektráren.
Poté co se určily a upravily pozemky pro stavbu, přišli experti s vážným varováním:
některé pozemky jsou příliš blízko sebe, což nebezpečně zvyšuje riziko řetězové reakce při výbuchu.
Dotace na stavbu ale byly již přiděleny, takže stavba všech elektráren musí být bezpodmínečně dokončena,
výbuch nevýbuch.

Naštěstí lze nebezpečí ještě trochu snížit. Elektrárny jsou totiž dvou typů: jedny pracují
se světlou energií a druhé s temnou energií. A světlá energie s temnou nijak neinteraguje, takže
postavit blízko sebe dvě elektrárny odlišných typů není nijak nebezpečné.

% This is how you can include a picture:

% \includegraphics[height=5cm]{img/picture.jpg}

\heading{Úloha}

Na vstupu jsou dány pozice $n$ bodů (pozemků pro elektrárny) v rovině (na povrchu planety).
Vaším úkolem je pro každý pozemek určit, zda zde má být postavena elektrárna pracující se světlou energií, nebo s temnou, a to tak, aby minimální vzdálenost mezi dvěma elektrárnami stejného typu byla co největší.
Tato vzdálenost bude rovněž součástí výstupu, a abychom se vyhnuli iracionálním číslům na výstupu,
budeme po vás chtít vypsat její druhou mocninu, která je vždy celým číslem.

\heading{Vstup}

Na prvním řádku vstupu je celé číslo $N$ ($3 \leq N \leq 100\,000$) -- počet pozemků.
Následuje $N$ řádků, z nichž $i$-tý obsahuje celá čísla $x_i$, $y_i$ ($0 \leq x_i, y_i \leq 10^9$)
-- souřadnice $i$-tého pozemku. Zadané body jsou po dvou různé.

\heading{Výstup}

Na první řádek výstupu vypište druhou mocninu největší možné minimální vzdálenosti.
Na druhý řádek vypište počet elektráren pracujících se světlou energií a na třetí řádek mezerami oddělený
seznam jejich čísel (elektrárny jsou číslovány od jedné dle pořadí na vstupu).
Podobně na čtvrtý řádek vypište počet elektráren pracujících s temnou energií a na pátý řádek jejich seznam.
Zřejmě je vždy více možných řešení; vypište kterékoli.

\heading{Podproblémy}

\begin{center}
\begin{tabular}{|l|l|l|}
\hline
Podproblém & Body & Nejvyšší $n$  \\ \hline
1       & 10     & 100           \\ \hline
2       & 25     & 2000         \\ \hline
3       & 65     & 100\,000       \\ \hline
\end{tabular}
\end{center}

\heading{Příklady}


\sampleIN
4
0 3
0 0
3 0
3 3
\sampleOUT
18
2
1 3
2
2 4
\sampleCOMMENT
\sampleEND



\end{document}
