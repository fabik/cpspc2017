\input sys/inputs.tex
\usepackage[czech]{babel}

\begin{document}

\bigheading{Největší společní dělitelé}

% \info{task_name}{infile}{outfile}{points}{timelimit}{memlimit}
% leave this values, if you are not interested
\info{gcds}{stdin}{stdout}{100}{2000 ms}{1 GB}



% This is how you can include a picture:

% \includegraphics[height=5cm]{img/picture.jpg}

\heading{Úloha}

Je dáno $n$ přirozených čísel $a_1,a_2,\dots ,a_n$. Rozdělíme je do $k$ skupin, pro každou skupinu spočítáme jejího
největšího společného dělitele a těchto $k$ výsledků sečteme.

Vaším úkolem je pro každé $k = 1, 2, \ldots, n$ určit, jakého největšího součtu lze takto dosáhnout.

\heading{Vstup}

Na prvním řádku vstupu je celé číslo $n$ ($1 \leq n \leq 500\,000$).
Na druhém řádku jsou celá čísla $a_1,a_2,\dots ,a_n$ ($1\leq a_i\leq 10^{12}$).

\heading{Výstup}

Vypište $n$ řádků a na $k$-tém z nich nechť je největší součet, kterého lze dosáhnout při dělení na $k$ skupin.

\heading{Podproblémy}

\begin{center}
  \begin{tabular}{|c|p{8cm}|c|}
    \hline
    \textbf{Podproblém} & \textbf{Omezení} & \textbf{Body} \\ \hline
    1 & $n \leq 7$ & 5 \\ \hline
    2 & $n \leq 15$ & 5 \\ \hline
    3 & $n \leq 100$, $a_i \leq 500$ & 8 \\ \hline
    4 & $n \leq 2000$, $a_i \leq 2000$, $a_i$ jsou po dvou různá & 8 \\ \hline
    5 & $n \leq 2000$ & 14 \\ \hline
    6 & $a_i$ jsou po dvou různá & 25 \\ \hline
    7 & žádná omezení & 35 \\ \hline
  \end{tabular}
\end{center}

\heading{Příklady}
\sampleIN
4
10 9 10 3
\sampleOUT
1
13
23
32

\sampleEND

\noindent \textit{Pro $k = 2$ je nejlepší rozdělení na skupiny $(10,10)$ a $(9,3)$, což dává součet $10+3 = 13$.
Pro $k=3$ je nejlepší rozdělení na skupiny $(10)$, $(10)$ a $(9,3)$, což dává součet $10+10+3 = 23$.}

\medskip

\sampleIN
8
15 25 29 30 43 44 45 55
\sampleOUT
1
56
101
145
188
221
256
286

\sampleEND



\end{document}
